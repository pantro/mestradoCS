\documentclass[10pt]{article}
\usepackage[utf8]{inputenc}
\usepackage[brazil]{babel}
\usepackage{ae}
\usepackage{amssymb,amsmath,amssymb,graphicx,fancyhdr,epsfig,psfrag,tabularx,float,caption}
\usepackage[paperwidth=210mm,paperheight=297mm]{geometry}
\usepackage{times}
\usepackage{textcomp}
\usepackage[table]{xcolor}
\usepackage{colortbl}
\usepackage{url}
\usepackage{listings} % inclusion of source code
\usepackage{hyperref} % clean URLs

% the following is needed for syntax highlighting
\usepackage{color}

\definecolor{dkgreen}{rgb}{0,0.6,0}
\definecolor{gray}{rgb}{0.5,0.5,0.5}
\definecolor{mauve}{rgb}{0.58,0,0.82}

\lstset{ %
  language=Java,                  % the language of the code
  basicstyle=\footnotesize,       % the size of the fonts that are used for the code
  numbers=left,                   % where to put the line-numbers
  numberstyle=\tiny\color{gray},  % the style that is used for the line-numbers
  stepnumber=1,                   % the step between two line-numbers. If it's 1, each line 
                                  % will be numbered
  numbersep=5pt,                  % how far the line-numbers are from the code
  backgroundcolor=\color{white},  % choose the background color. You must add \usepackage{color}
  showspaces=false,               % show spaces adding particular underscores
  showstringspaces=false,         % underline spaces within strings
  showtabs=false,                 % show tabs within strings adding particular underscores
  frame=single,                   % adds a frame around the code
  rulecolor=\color{black},        % if not set, the frame-color may be changed on line-breaks within not-black text (e.g. commens (green here))
  tabsize=4,                      % sets default tabsize to 2 spaces
  captionpos=b,                   % sets the caption-position to bottom
  breaklines=true,                % sets automatic line breaking
  breakatwhitespace=false,        % sets if automatic breaks should only happen at whitespace
  title=\lstname,                 % show the filename of files included with \lstinputlisting;
                                  % also try caption instead of title
  keywordstyle=\color{blue},          % keyword style
  commentstyle=\color{dkgreen},       % comment style
  stringstyle=\color{mauve},         % string literal style
  escapeinside={\%*}{*)},            % if you want to add a comment within your code
  morekeywords={*,...}               % if you want to add more keywords to the set
}

% \newcommand{\blue}{\textcolor{blue}}
% \newcommand{\blue}{}
\hyphenation{pro-ble-mas}

\begin{document}

\selectlanguage{brazil}

\begin{large}
	\begin{center}
		\textbf{MC302} \\
		Primeiro semestre de 2017 \\ \vspace{0.5cm}
		\textbf{Laboratório 1}
	\end{center}
\end{large}
\vspace{0.25cm}
\noindent \textbf{Professor(a):} Esther Colombini (esther@ic.unicamp.br) \\
\textbf{PEDs:} Elisangela Santos (ra149781@students.ic.unicamp.br), Lucas Faloni (lucasfaloni@gmail.com), Lucas David (lucasolivdavid@gmail.com), Wellington Moura (wellington.tylon@hotmail.com) \\
\textbf{PAD:} Túlio Martins (tuliomartinstm1523@gmail.com) \\
%\noindent \textbf{Professor(a):} Fábio Luiz Usberti (fusberti@ic.unicamp.br) \\
%\textbf{PEDs:} Natanael Ramos (naelr8@gmail.com), Rafael Arakaki (rafaelkendyarakaki@gmail.com) \\
%\textbf{PAD:} Bleno Claus (blenoclaus@gmail.com) \\

\hrule \hrule

\section{Objetivo}

O objetivo desta atividade consiste na familiarização com o ambiente de desenvolvimento integrado (IDE, \textit{Integrated Development Environment}) chamado Eclipse\footnote{https://eclipse.org} e a linguagem de programação Java\footnote{https://www.java.com}.

\section{Atividade}

Nesta atividade o principal foco será a familiarização com o Eclipse e a programação de duas classes chamadas \textbf{CartaLacaio} e \textbf{CartaMagia}. A primeira tarefa será configurar o ambiente com a criação de um novo projeto e de uma nova classe para então programar.

Os seguintes passos podem ser tomados para a criação do projeto:
\begin{enumerate}
    \item Abra o Eclipse.
    \item Crie um novo projeto (File -> New -> Java Project).
    \item Digite o nome do projeto (ex: Lab1).
    \item Na aba JRE escolha a última versão do JavaSE instalado na máquina (ex: JavaSE-1.7 ou JavaSE-1.8).
    \item Clique em Finish.
\end{enumerate}

Para criar uma nova classe faça:
\begin{enumerate}
	\item Utilize a aba 'Package Explorer' que aparece do lado esquerdo da IDE.
    \item Crie uma nova classe no projeto (Botão direito no projeto -> New -> Class).
    \item Digite o nome da classe.
    \item Programe a classe.
\end{enumerate}

\section{Classe CartaLacaio}

A classe CartaLacaio deste laboratório é baseada em um jogo de cartas de computador chamado Hearthstone\footnote{http://us.battle.net/hearthstone/pt} \textcopyright, neste jogo existem cartas do tipo \emph{Lacaio} que possuem atributos como ataque e vida distintos para cada carta.

A classe CartaLacaio deve ter os seguintes atributos:
\begin{itemize}
    \item ID (número inteiro)
    \item nome (cadeia de caracteres - \emph{String})
    \item ataque (número inteiro)
    \item vidaAtual (número inteiro)
    \item vidaMaxima (número inteiro)
    \item custoMana (número inteiro)
\end{itemize}

O exemplo abaixo apresenta a declaração da classe CartaLacaio e seus atributos. Note que todas as variáveis são declaradas como privadas (\textbf{private}). Note também a implementação dos métodos de acesso get() e set(), esses métodos são comumente utilizados na linguagem Java para acessar os atributos dos objetos.

\lstinputlisting[language=Java]{CartaLacaio.java}

Além disso a classe CartaLacaio deve conter um método construtor, o método construtor deve receber como argumentos os atributos para inicializar o objeto. Para ilustrar esse conceito melhor, veja o exemplo abaixo.

\lstinputlisting[language=Java]{MetodoConstrutor.java}

Também é necessário que a classe CartaLacaio possua uma função \textbf{toString()} que devolve uma String contendo uma descrição geral dos atributos da carta. Veja o exemplo abaixo:

\lstinputlisting[language=Java]{toString.java}

Observe que são utilizados os métodos de get e set. É necessário programar estes métodos antes de utilizá-los, logo para cada atributo da classe CartaLacaio deve existir um método get e set correspondente. O formato do método toString() a ser implementado é livre, mas todos os atributos devem ser impressos.

Faça a implementação do método \textbf{construtor}, métodos \textbf{get()} e \textbf{set()} de todos os atributos e do método \textbf{toString()} para a classe \textbf{CartaLacaio}.

\section{Classe CartaMagia}

Utilizando a classe CartaLacaio como base, crie a classe CartaMagia e faça a implementação do método \textbf{construtor}, métodos \textbf{get()} e \textbf{set()} de todos os atributos e do método \textbf{toString()} para a classe \textbf{CartaMagia}.

A classe CartaMagia deve ter os seguintes atributos:
\begin{itemize}
    \item ID (número inteiro)
    \item nome (cadeia de caracteres - \emph{String})
    \item dano (número inteiro)
    \item area (variável da lógica booleana - \emph{boolean})
    \item custoMana (número inteiro)
\end{itemize}

\section{Classe Main}

Para um programa Java funcionar é requerida a existência de um método main que serve de ponto da partida para o programa ser inicializado. Crie uma nova classe através do Eclipse chamada Main e escolha a opção para gerar automaticamente o método main.

Na função main realize a instanciação de alguns objetos do tipo CartaLacaio ou CartaMagia (pelo menos um de cada) com valores de atributos quaisquer conforme sua imaginação. Após instanciar os objetos, imprima seus dados utilizando o método System.out.println(). Veja o exemplo a seguir:

\lstinputlisting[language=Java]{print.java}

Observe que ao imprimir os dados dos objetos da classe CartaLacaio ou CartaMagia, o método toString() que você implementou foi chamado implicitamente. 

Após implementar as três classes, para executar o programa e ver o resultado clique no botão ``Run'' do Eclipse.

\section{Tarefas}

\begin{itemize}
	\item Criação do projeto e classes.
	\item Programação dos métodos construtores das classes CartaLacaio e CartaMagia.
	\item Programação dos métodos get e set das classes CartaLacaio e CartaMagia.
	\item Programação dos métodos toString das classes CartaLacaio e CartaMagia.
	\item Programação do método main e impressões de algumas cartas.
\end{itemize}

\section{Submissão}

Para submeter a atividade utilize o Moodle (\url{https://www.ggte.unicamp.br/ea}). Salve os arquivos dessa atividade em um arquivo comprimido no formato .tar.gz e nomeie-o \textbf{Lab1-000000.tar.gz} trocando '000000' pelo seu número de RA. Submeta o arquivo na seção correspondente para esse laboratório no moodle da disciplina MC302.

\textbf{Datas de entrega}
\begin{itemize}
    \item Dia \textbf{20/03} Turma \textbf{ABCD} até às 23:55
    \item Dia \textbf{17/03} Turma \textbf{EF} até às 23:55
\end{itemize}

\end{document}
% end of file template.tex